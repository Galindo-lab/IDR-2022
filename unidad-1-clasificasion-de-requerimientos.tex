% Created 2022-08-22 lun 12:14
% Intended LaTeX compiler: pdflatex
\documentclass[11pt]{article}
\usepackage[utf8]{inputenc}
\usepackage[T1]{fontenc}
\usepackage{graphicx}
\usepackage{grffile}
\usepackage{longtable}
\usepackage{wrapfig}
\usepackage{rotating}
\usepackage[normalem]{ulem}
\usepackage{amsmath}
\usepackage{textcomp}
\usepackage{amssymb}
\usepackage{capt-of}
\usepackage{hyperref}
\author{Galindo}
\date{\today}
\title{Tipos De Requerimientos}
\hypersetup{
 pdfauthor={Galindo},
 pdftitle={Tipos De Requerimientos},
 pdfkeywords={},
 pdfsubject={},
 pdfcreator={Emacs 26.3 (Org mode 9.1.9)}, 
 pdflang={English}}
\begin{document}

\maketitle
\tableofcontents


\section{Funcional}
\label{sec:org4b417dd}
Definición de los servicios que el sistema debe proporcionar, cómo debe reaccionar a una entrada particular y cómo se debe comportar ante situaciones particulares. los requerimientos se dividen en dos tipos principales:

\begin{description}
\item[{de usuario}] son declaraciones de los servicios que se espera que el sistema provea y de las restricciones bajos las cuales se debe operar.

\item[{del sistema}] establecen con detalle los servicios y restricciones del sistema.
\end{description}

\section{No funcionales}
\label{sec:orga9a69e7}
Son los que actúan para limitar la soluciones, se los conoce como restricciones o requisitos de calidad. A los requisitos no funcionales se los puede dividir en:

\begin{description}
\item[{Requisitos de producto}] Estos especifican el comportamiento del producto.

\item[{Requisitos de organización}] Se derivan de políticas y procedimientos existentes en la organización del cliente y en la del desarrollador.

\item[{Requisitos externos}] Son los requisitos que derivan de los factores externos al sistemas y de su proceso de desarrollo, incluyen requerimientos de inseparabilidad que definen la manera en que el sistema interactúa con los otros sistemas de la organización.
\end{description}

\section{FURPS+}
\label{sec:org2a868e4}
Patrones de clasificación de requerimientos

\begin{description}
\item[{Facilidad de uso (Usability)}] Características, capacidades y algunos aspectos de seguridad, por ejemplo: se debe ver el texto fácilmente a una distancia de un metro

\item[{Fiabilidad (reliability)}] Factores humanos (interacción), documentación, ayuda, etc, un ejemplo de este tipo de requerimiento sería: Si se produce algún fallo al usar un servicio externo solucionarlo localmente.

\item[{Rendimiento (performance)}] Tiempo de respuesta, productividad, precisión, disponibilidad, use de los recursos. Ejemplo: Conseguir la autorización de pago en menos de 1 minuto el 90\% de las veces.

\item[{Soporte (Supportability)}] Adaptabilidad, facilidad de mantenimiento, internacionalización, facilidad de recuperación de un fallo y grado de previsión. Ejemplo: el sistema debe ser instalable por los usuarios.

\item[{Implementación (plus)}] Limitación de recursos, lenguajes y herramientas, hardware.

\item[{Interfaz (plus)}] Restricciones impuestas para la interacción con sistemas externos (no es GUI).

\item[{Operaciones (plus)}] Gestión del sistema, pautas administrativas, puesta en marcha.

\item[{Empaquetamiento (plus)}] Forma de distribución.

\item[{Legales (plus)}] Licencia, derechos de autor, etc.
\end{description}
\end{document}
